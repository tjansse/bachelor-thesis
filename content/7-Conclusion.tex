% !TeX encoding=unicode
% !TeX spellcheck = de-DE

\chapter{Conclusion}
... 

Version 2.0 of \mcgrid{} will allow the creation of grids for calculations matching NLO QCD computations and parton showers using \mcatnlo{}.
This has not been verified yet for Higgs production.
However, as has been seen in section \textcolor{red}{???} the \mcatnlo{} approach allows for a more accurate description of soft and collinear emissions than fixed order NLO calculations.
Therefore ...


In section \textcolor{red}{???} we examined, how \appl{} and \fnlo{} parametrize the $x$-distribution to provide a better coverage of the data on a grid with equal-sized bins.
Which transformation should be used depends on the considered observable.
To achieve the best possible performance, one would have to check the actual $x$-distribution in each case.
The provided transformations are obviously a compromise to cover the needs of a large number of processes.
The drawback of this approach is, that for some observables one needs unnecessary large grids to reliably reproduce them.
There is, however, an alternative way: When performing the phasespace run prior to the fill run, one could, instead of only determining the limit values of $x$ and $Q^2$, sample the whole distribution.
By the use of numerical inversion, the data could then be used to provide a transformation that represents the actual distribution of $x$ or $Q^2$ in the process.
Thus, the bins of the grid would be filled ideally and a smaller grid would suffice to reproduce the desired observable.
This would improve the overall performance of the software, as a smaller grid means less computation time and less memory consumption.

