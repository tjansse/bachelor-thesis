% !TeX encoding=unicode
% !TeX spellcheck = de-DE

\chapter{Conclusion}
In this thesis, an interpolation method has been used to accelerate the variation of parameters in a fixed order NLO calculation.
\mcgrid{} has been used to interface two different tools, \appl{} and \fnlo{}, which have been used to fill interpolation grids with the correct information.
This has been performed explicitly for the example of Higgs boson production at the LHC.
It could be verified that the applied method is able to achieve sub-percent accuracy in the reproduction of different observables.
The possibility of \textit{a posteriori} variation of the renormalization and factorization scales has been demonstrated.
It shows no significant decrease in accuracy when no additional hard jets are considered but problems in the reproduction emerged in the cases of one and two additional jets (cf. \cref{sec:scalesvar}).
They seem to be caused by the technique that is used to variate the scales and not by errors in the interpolation method itself.
Future investigations will isolate the cause of the problem and deduce what is needed to fix it.
Changing the PDF during the reevaluation of the cross section, however, has caused no further problems and has provided reproduction accuracy at permille level.

Two examples have been demonstrated that give an impression of possible applications.
Global PDF fits benefit from the possibilty of evaluating a large number of PDFs in cross section calculations.
Without tools like \appl{} and \fnlo{}, many processes could not be included in fits beyond LO.
In addition to that, the interpolation tools simplify the treatment of uncertainties in theoretical predictions based on Monte Carlo event generators.

Further studies concerning the use of \mcgrid{} in the Higgs production process could include dynamically allocated scales.
This case adds further complexity to the interpolation grids as the dependence on the scales has to be interpolated as well.
It could not be validated yet.
In a next step the support of the fixed order expansion of \mcatnlo{}, which is introduced in version 2.0 of \mcgrid{}, needs to be verified as well.
That would pave the way for a complete inclusion of multijet merged calculations.
It is a desirable goal, because merged calculations are very complex and time consuming and therefore are very well suited for a method that allows faster computation.
%... 
%
%Version 2.0 of \mcgrid{} will allow the creation of grids for calculations matching NLO QCD computations and parton showers using \mcatnlo{}.
%This has not been verified yet for Higgs production.
%However, as has been seen in section \textcolor{red}{???} the \mcatnlo{} approach allows for a more accurate description of soft and collinear emissions than fixed order NLO calculations.
%Therefore ...

In \cref{sec:xtransform} we examined how \appl{} and \fnlo{} parametrize the $x$\-/distribution to provide a better coverage of the data on a grid with equal-sized bins.
Which transformation should be used depends on the considered observable.
To achieve the best possible performance, one would have to check the actual $x$-distribution in each case.
The provided transformations are obviously a compromise to cover the needs of a large number of processes.
The drawback of this approach is that for some observables one needs unnecessary large grids to reliably reproduce them.
There is, however, an alternative way: When performing the phasespace run prior to the fill run one could, instead of only determining the limit values of $x$ and $Q^2$, sample the whole distribution.
By the use of numerical inversion, the data could then be used to provide a transformation that represents the actual distribution of $x$ or $Q^2$ in the process.
Thus, the bins of the grid would be filled ideally and a smaller grid would suffice to reproduce the desired observable.
This would improve the overall performance of the software, as a smaller grid means less computation time and less memory consumption.

