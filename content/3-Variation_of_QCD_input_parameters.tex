% !TeX encoding=unicode
% !TeX spellcheck = de-DE

\chapter{Variation of QCD input parameters}
\label{ch:parameter_variation}
%
\section{Reweighting QCD calculations}
Often it is needed to vary the parameters in QCD calculations, for example the scale variables and PDFs to estimate the uncertainty.
When the number of variations becomes large, it is not practical to rerun the whole event generation for each calculation as the time and resource consumption become too high.
Instead it is possible to reuse information from previously generated events and combine them with the new parameters.
Here we consider an NLO calculation using Catani-Seymour subtraction \cite{catani_seymour1997}, which is the scheme implemented in \sherpa{} and \mcgrid{}.
Other subtraction schemes can be used as well, as is demonstrated in \cite{amcfast} for the FKS scheme \cite{fks_a,fks_b}.

The total cross section can be separated into four finite terms:
%
\begin{equation}
	\sigma^\text{NLO}_{pp \rightarrow X} = \int \dif \hat \sigma^B + \int \dif \hat \sigma^V + \int \dif \hat \sigma^I + \int \dif \hat \sigma^{RS} \,
\end{equation}
%
They correspond to the Born (B), virtual (V), integrated subtraction (I) and real subtraction (RS) parts of the calculation.
We need to work out the explicit dependence on the parameters to accurately reweight them.

For the B and RS terms this is straightforward as they resemble LO calculations.
Making use of the factorization theorem we can write them in the form (cf. \cref{eq:hadron_crosssection})
%
\begin{equation}
	\sigma^{B/RS} = \sum_{i,j} \int \dif x_1 \dif x_2 \int \dif \Phi_n \left( \frac{\alpha_s(\mu_R^2)}{2 \pi} \right)^p \pdf_i(x_1,\mu_F^2) \pdf_j(x_2,\mu_F^2) \dif \hat\sigma \, ,
\end{equation}
%
where for the B term $p = p_\text{LO}$ and for the RS term $p = p_\text{LO} + 1$.
Using Monte Carlo integration, we can rewrite this as a sum over generated events:
%
\begin{equation}
  \int \dif \hat \sigma^{B/RS} = \sum_{e=1}^{N_\text{evt}} \alpha_s^p(\mu_R^2) \pdf_1(x_1,\mu_F^2) \pdf_2(x_2,\mu_F^2) w_e^{(0)} \, ,
  \label{eq:reweight_lo}
\end{equation}
%
where the weight $w_e^{(0)}$ contains the parton-level matrix element squared.

The virtual part occupies a more complicated structure because the weights explicitely depend on the renormalization scale.
It takes the form
%
\begin{equation}
	\int \dif \hat \sigma^V = \sum_{e=1}^{N_\text{evt}} \alpha_s^{p_\text{LO} + 1}(\mu_R^2) \pdf_1(x_1,\mu_F^2) \pdf_2(x_2,\mu_F^2) \left\{ w_e^{(0)} + l_R w_e^{(1)} + l_R^2 w_e^{(2)} \right\} \, ,
\end{equation}
%
with renormalization scale logarithms $l_R = \log\left(\frac{\mu_r^2}{\mu_{R,0}^2} \right)$, where $\mu_{R,0}$ is the scale at which $w_e^{(1)}$ and $w_e^{(2)}$ were originally evaluated.

In the I part, the event weights have a complex dependence on the PDF.
The full structure can be written as
%
\begin{align}
	\int \dif \hat \sigma^I = \sum_{e=1}^{N_\text{evt}} &\alpha_s^{p_\text{LO} + 1}(\mu_R^2) \left\{ \vphantom{\left( \sum_{k=1}^4 \right)} f_1(i,x_1,\mu_F^2) w_e^{(0)} f_2(j,x_2,\mu_F^2) \right. \nonumber \\
								&+ \left( \sum_{k=1}^4 f_1^{(k)}(i,x_1,x'_1,\mu_F^2) w_{e,k}^{(3)} \right) f_2(j,x_2,\mu_F^2) \nonumber \\
								&\left. + f_1(j,x_1,\mu_F^2) \left( \sum_{k=1}^4 w_{e,k}^{(4)} f_2^{(k)}(j,x_2,x'_2,\mu_F^2) \right) \right\} \, ,
\end{align}
%
where the $x$ and $x'$ are the Björken-$x$ before and after initial state branching.
Moreover, $i$ and $j$ denote the parton flavours before the branching and the $f^{(k)}$ provide the correct form of the PDF depending on the type of branching.
They are given explicitly in \cite{mcgrid2013}.

We can dispose of the weights $w^{(1)}$ and $w^{(2)}$ by chosing a central scale.
If we furthermore project the contributions from $w^{(3)}$ and $w^{(4)}$ onto separate events, we can write the total NLO cross section in the compact form
%
\begin{equation}
  \sigma^\text{NLO}_{pp \rightarrow X} = \sum_{e=1}^{N_\text{evt}} \alpha_s^{p_e}(\mu_R^2) \pdf_1(x_1,\mu_F^2) \pdf_2(x_2,\mu_F^2) w_e \, ,
  \label{eq:reweight_nlo}
\end{equation}
%

Provided that all event weights have been stored, it is now possible to change the values of $\alpha_s$, $\mu_R$ and $\mu_F$ or use a different PDF \textit{a posteriori}.
%
\section{Interpolation grids}
Despite the advantage towards regenerating all events for the variation of a single parameter, the reweighting approach is still not a satisfying solution for many use cases.
The whole event record has to be stored, which can easily reach many gigabytes or even terabytes in high statistics computations.
The storing and reprocessing of the events may be a challenge by itself and is not convenient if more than a few parameter variations have to be performed.
One therefore whishes to somehow decrease the resource requirements without losing a significant amount of accuracy.
In the ideal case, it should be mostly independent on the statistics.
A possible solution is the use of interpolating grids to represent the PDFs and event weights.
Then only a uniquely defined number of values has to be saved, while values in-between the grid points are generated by interpolating functions.
Such a method has been implemented by the \appl{} \cite{applgrid2010} and \fnlo{} \cite{fastnlo2006,fastnlo2011} projects.

The PDF $\pdf(x,Q^2)$ depends on the momentum fraction $x$ and the momentum transfer $Q^2$.
It can be approximated as a sum over discrete grid points using interpolation functions $I$ of order $N$:
%
\begin{equation}
	\pdf(x,Q^2) = \sum_{i=0}^{N_x} \sum_{j=0}^{N_Q} \pdf(x_i,Q_j^2) I_i^{(N_x)}(x_1) I_j^{(N_Q)}(Q^2) \, .
\end{equation}
%
Consequently, we can write \cref{eq:reweight_nlo} as
%
\begin{alignat}{3}
  \sigma^\text{NLO}_{pp \rightarrow X}	&= \sum_{e=1}^{N_\text{evt}} \sum_{i,j=0}^{N_x} \sum_{k=0}^{N_Q} &&\alpha_s^{p_e}(Q_k^2) \pdf_1(x_i,Q_k^2) \pdf_2(x_j,Q_k^2) w_e \nonumber \\
  										&	&&\cdot I_i^{(N_x)}(x_1) I_j^{(N_x)}(x_2) I_k^{(N_Q)}(Q^2) \nonumber \\
  										&= \mathrlap{\sum_p \sum_{i,j=0}^{N_x} \sum_{k=0}^{N_Q} \alpha_s^p(Q_k^2) \pdf_1(x_i,Q_k^2) \pdf_2(x_j,Q_k^2) W_{i,j,k}^{(p)} \, ,}
\end{alignat}
%
where $p$ is the perturbative order and the interpolated weights $W_{i,j,k}^{(p)}$ are processed as a sum over the events:
\begin{equation}
	W_{i,j,k}^{(p)} = \sum_{e=1}^{N_\text{evt}} \delta_{p,p_e} w_e I_i^{(N_x)}(x_1) I_j^{(N_x)}(x_2) I_k^{(N_Q)}(Q^2) \, .
\end{equation}
%
Now the sum over the events has been completely absorbed into the definition of the interpolated weights.
These can be calculated in a single run of the event generator and be stored efficiently.
The calculation of the cross section has become a simple sum over the grid points and thus is much faster for a large number of events.
The approach can easily be extended to histogrammed data like differential cross sections by defining the observable bins and computing one weight $W_{i,j,k}^{(p),(s),(b)}$ for each bin $b$.
However, unlike the full event record, the weight grid is restricted to the observable it was constructed for and cannot be used to calculate any other values.

While the variation of the PDF and $\alpha_s$ is still straightforward, scale variation is slightly more complicated because the weights themselves depend on the scale choices.
Nevertheless, it is possible to vary the scales without too much effort, using DGLAP evolution.
The precise procedure is explained in \cite{applgrid2010}.
