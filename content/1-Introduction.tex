% !TeX encoding=unicode
% !TeX spellcheck = en-US

\chapter{Introduction}
Run 1 of the Large Hadron Collider (LHC) displayed great performance of the collider and its detectors.
It enabled high precision tests of theoretical predictions at unprecedented center of mass energies.
Simultaneously it encouraged theorists to expend huge effort in minimizing the uncertainties of calculations in quantum chromodynamics (QCD).
A great amount of progress could be achieved, among others the automation of next-to-leading order (NLO) calculations, the merging of parton shower results with different jet multiplicities and the matching of parton showers to NLO-accuracy calculations \cite{eventgenerators}.
As of 3 June this year, the LHC is delivering data again at an even higher center of mass energy of \SI{13}{\tera\electronvolt}.
The new run promises new results and an increase in statistics, allowing for a more detailed study of many observations.

Precise predicting of the expected results is crucial in a test of our current models.
To achieve this, uncertainties in the calculations have to be as small as possible.
However, with rising complexity the calculations become lengthy.
Single computations lasting for days are not uncommon even at NLO.
This is aggravated by the fact that they have to be repeated with different values of the coupling parameter $\alpha_s$, at different scales and with different parton distribution functions (PDFs) in order to get a reasonable estimate of the uncertainty.
Including PDF uncertainties is especially expensive as it involves a large number of repeated calculations.
PDF fitting groups, who include many different processes in their fits to the data, suffer from this particularly.
Without any speedup, their task would be impossible.
For this reason, methods have been developed to accelerate the process of varying the parameters, obviating the need for explicit recalculations.
One method is based on a special event file format, that stores all information required to recompute NLO observables with different parameters \cite{ntuples}.
A different approach is the on-the-fly reweighting that has been introduced in the 2.2.0 release of the Monte Carlo event generator \sherpa{} \cite{mcgrid20}.
The fastest currently possible reevaluations are achieved with interpolation tools like \appl{} \cite{applgrid2010} and \fnlo{} \cite{fastnlo2006}.
Two interfaces exist that allow the automated creation of the related interpolation grids in NLO calculations, namely \amcfast{} \cite{amcfast} and \mcgrid{} \cite{mcgrid2013}.

Currently, \mcgrid{} supports calculations at LO, NLO and the fixed-order expansion of \mcatnlo{}.
One of the goals of the project is the support of multijet merging calculations.
Along this road, it has to be validated that the interpolation method works correctly when the different jet multiplicities are treated separately.
In this thesis, the process of Higgs boson production through gluon fusion in combination with different jet multiplicities is considered.
It had not been verified yet, that it is supported by \mcgrid{}.
After the discovery of the Higgs boson \cite{higgsdiscovery_atlas2012,higgsdiscovery_cms2012}, one of the tasks of the \SI{13}{\tera\electronvolt} LHC will be the determination of its properties.
Increased statistics will allow the study of differential cross sections for the first time.
It is therefore important to enable the detailed analysis of theoretical uncertainties in this process.

This thesis is organized as follows.
First, in \cref{ch:pqcd,ch:parameter_variation}, an overview of the theoretical foundations involved in the calculations is given.
It is followed by an illustration of the gluon fusion process in \cref{ch:gfusion}.
There, a brief study of the Higgs transverse momentum in perturbative QCD is included.
In \cref{ch:validation} the interpolation method implemented by \mcgrid{} is validated for the gluon fusion process with different jet multiplicities.
Proof is provided that it is possible to vary the QCD parameters \textit{a posteriori} using either \appl{} or \fnlo{}.
Two application examples are presented in \cref{ch:examples}.
They are supposed to demonstrate the usefulness of the method.
The thesis finishes with a conclusion and an outlook on possible extensions in the future.
